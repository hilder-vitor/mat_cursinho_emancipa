% Aula sobre Sistemas Lineares
\documentclass[a4paper,12pt]{article}
\usepackage[brazilian]{babel}
\usepackage[utf8]{inputenc}


\usepackage{amsthm}
\usepackage{graphicx}
\usepackage{amssymb}  % para o simbolo dos reais e outros

% --------  Fim da seção de parâmetros globais

\begin{document}    % Começo do conteúdo.

\title{Matemática}
\author{Vitor Lima (vitor\_lp@yahoo.com)}

\maketitle

%%%%      %%%%%%%%%%%%%%%%%%	   %%%%%%%%%
%%%%%%%%%%%%%%%%%%%%%%%%%%%%%%%%%%%%%%%%%%%%%%%%%%%%
%%%%%%%%%%%% Sistemas Lineares  	%%%%%%%%%%%%%%%%
%%%%%%%%%%%%%%%%%%%%%%%%%%%%%%%%%%%%%%%%%%%%%%%%%%%%
%%%%%%%%%%%%%%%		%%%%		%%%%%%%%%%%%

\section{Introdução}
    \emph{Digamos que em uma prova com 50 questões, cada resposta correta valha 3 pontos e cada questão errada acarrete em uma perda de 2 pontos. \\
    Se alguém obteve 75 pontos nesta prova, quantas questões ele acertou ?} \\ \\
    Este é um problema típico de sistemas lineares, pois envolve mais de um valor desconhecido (neste caso, a quantidade de respostas corretas e a quantidade de respostas erradas). \\
    Estudaremos agora, quando é possível resolver este tipo de problema, como resolver, veremos alguns exemplos e exercícios.



\section{Definição}
    Um sistema de equaçõs lineares é um conjunto finito de equações lineares, cada uma com um número finito de variáveis (ou incógnitas). \\
    Geralmente, um sistema é representado utilizando uma chave do lado esquerdo e colocando-se um equação por linha. Por exemplo:\\
    
$ 
\left\{
\begin{array}{ll}
\displaystyle 3x + y  = 10 \\
\displaystyle x + y + z = 0 \\
\displaystyle y - z = -5
\end{array}
\right.\\
$
   \\  \\ Este é um sistema linear com três equações e três incógnitas (\emph{x, y} e \emph{z}). \\
    As soluções de um sistema linear são os valores que satisfazem todas as equações simultâneamente. Assim, para o sistema acima, temos como solução a \emph{tripla} $(x, y, z) = (5, -5, 0)$.\\
    Note que estes valores satisfazem todas as equações (por exemplo, na primeira equação, $3\cdot5 - 5 = 15 - 5 = 10$).



\section{Como resolver}
    Nesta seção, vamos aprender duas formas de resolver sistemas lineares:
    
    \subsection{Método da Substituição}
    Podemos resolver um sistema linear isolando uma variável em uma equação e substituindo-a nas outras equações. Cada vez que substituimos uma variável, ela desaparece das outras equações. Assim, repetimos isto (vamos isolando as próximas e substituindo nas equações restantes) até ficarmos só com uma variável, pois neste ponto teremos uma equação linear simples e saberemos resolver. \\
    Por exemplo, considere o seguinte sistema de três variáveis:
    
$ 
\left\{
\begin{array}{ll}
\displaystyle x + y + z = 0 \\
\displaystyle 4x + 4y  + 2z = 5 \\
\displaystyle  x -y + z = 5
\end{array}
\right.\\
$
   \\
   Isolando a variável $x$ na primeira equação, obtemos $x = -y - z$. Então, substituimos $x$ nas duas outras equações: \\
   Na segunda: $4(-y - z) + 4y + 2z = 5$, mas esta equação equivale à $-2z = 5$ \\
   Na terceira: $(-y -z) -y + z = 5$, mas esta equivale à $-2y = 5$ \\
    Assim, ficamos com o sistema \\
$ 
\left\{
\begin{array}{ll}
\displaystyle x + y + z = 0 \\
\displaystyle -2z = 5 \\
\displaystyle -2y = 5 \\
\end{array}
\right.\\
$
   \\
 
 Da segunda equação, vemos que $z = -\frac{5}{2}$ e, da terceira, vemos que $y = -\frac{5}{2}$.\\
 Agora, para descobrir o valor de $x$, substituimos os valores encontrados para $y$ e $z$ na primeira equação: \\
  $x -\frac{5}{2}  -\frac{5}{2} = 0 \Rightarrow x = 5$ \\
  Assim, a solução deste sistema é
  $$(x, y, z) = (5, -\frac{5}{2}, -\frac{5}{2}) $$
   
     \subsection{Método da Adição}
     Este método de resolução também é conhecido como \emph{Método de escalonamento}. Ele consiste em multiplicar uma equação por algum valor (já iremos discutir como escolher este valor) e depois somar as equações. \\
     Por exemplo, considere o sistema: \\
   $ 
\left\{
\begin{array}{ll}
\displaystyle x + 4y = 100 \\
\displaystyle 2x + 3y = 90 \\

\end{array}
\right.\\
$
   \\
   
   Podemos multiplicar a primeira equação por $-2$, obtendo:
   $$-2x -8y = -200$$
   agora, se somarmos esta equação com a segunda, teremos:
   
   $ 
\left\{
\begin{array}{ll}
\displaystyle -2x -8y = -200 \\
\displaystyle 2x + 3y = 90 \ \ \  \  \oplus\\
\displaystyle ----------- \\
\displaystyle 0 -5y = -110 \\
\end{array}
\right.\\
$
   \\
   (Somamos $x$ com $x$, $y$ com $y$ e valores sem variáveis com valores sem variáveis: $-2x + 2x$, $-8y + 3y$ e $-200 + 90$).\\
   Assim, a equação resultante nos diz que
   $$y = \frac{-110}{-5} = 22.$$
   Para obtermos o valor de $x$, substituimos este valor de $y$ na primeira equação: \\
   $$x + 4\cdot22 = 100  \Rightarrow x = 100 - 4\cdot22 = 12$$
   \emph{Observações: Como escolher por qual valor multiplicar a equação?}\\
   O objetivo é fazer uma das variáveis desaparecer ao somarmos as equações, então, devemos olhar para as duas equações que vamos somar e descobrir qual valor fará com que uma variável suma. \\
   Ou seja, escolha duas equações que serão somadas, escolha uma variável para desaparecer e depois, multiplique uma das equações por um valor que faça o coeficiente (número que está multiplicando a variável na equação) da variável em uma equação ficar igual ao negativo na outra equação. \\
   Se em uma equação houver algo como $ax$, onde $a$ é um valor qualquer, e em outra equação houver algo como $x$, podemos simplesmente multiplicar a segunda por $-a$. \\
  
  
\section{Classificação}
    Aprendemos como resolver um sistema linear, mas, devemos saber que um sistema linear nem sempre tem solução, ou as vezes, possui infinitas soluções. \\
    Apresentaremos agora uma calssificação de sistemas lineares baseada no número de soluções que ele apresente: \\
    \subsection{Impossível}
    Um sistema linear é dito impossível se ele não possuir soluções. Por exemplo, o sistema \\
   $ 
\left\{
\begin{array}{ll}
\displaystyle -2x -y = -2 \\
\displaystyle 2x + y = -2 \\
\end{array}
\right.\\
$
   \\
   não possui solução. Para ver isto, note que se multiplicarmos a primeira equação por $-1$, obteremos $2x + y = 2$, enquanto a segunda equação diz que $2x + y = -2$. Ou seja, a soma de $y$ com o dobro de $x$ deve ser ao mesmo tempo 2 e $-2$. \\
   Se tentarmos resolver este sistema, chegaremos em algo como $0 = -4$, o que é uma igualdade falsa. Isto é mais uma prova de que este sistema não tem solução.
   
    \subsection{Possível}
    Dizemos que um sistema é possível se ele possuir ao menos uma solução. \\
    Existem dois tipos de sistemas possíveis:
    \begin{enumerate}
    \item \textbf{Possível e determinado:} tem exatamente uma solução. Com exceção do último exemplo, todos os outros visto até então eram assim.
    
    \item \textbf{Possível e indeterminado:} tem infinitas soluções. \\
        Um exemplo de sistema possível e indeterminado é: \\
        $ 
\left\{
\begin{array}{ll}
\displaystyle x - y + z = 0 \\
\displaystyle y - z = 0\\
\end{array}
\right.\\
$
   \\
   Da segunda equação encontramos $y = z$. Substituindo na primeira, ficamos com $x - y + y = 0$, o que implica $x = 0$.\\
   Assim, todas as triplas da forma $(x, y, z) = (0, \alpha, \alpha)$, com $\alpha$ sendo qualquer valor, são soluções ($x = 0$, $y = z = 1$ é solução; $x = 0$, $y = z = 2$ também é solução, $x = 0$, $y = z = 2.5$ também é solução, etc...)
        
    \end{enumerate}
	
\section{Exercícios}

\begin{enumerate}


    \item Resolva os seguintes sistemas lineares:
    \begin{enumerate}
    \item
            $ 
\left\{
\begin{array}{ll}
\displaystyle 2x -y + 3z = 0 \\
\displaystyle 2y -z = 1\\
\displaystyle 2z = -6\\
\end{array}
\right.\\
$
   \\    
   
   \item 
       $ 
\left\{
\begin{array}{ll}
\displaystyle 3x -2y + z = 2 \\
\displaystyle y - z = 0\\
\end{array}
\right.\\
$
   \\    


   \item 
       $ 
\left\{
\begin{array}{ll}
\displaystyle 3x_1 - 5x_2 = 6 \\
\displaystyle 2x_2 = 2\\
\end{array}
\right.\\
$
   \\    
   \end{enumerate}


    %%% RESP: C
    \item O sistema abaixo \\
        $ 
\left\{
\begin{array}{ll}
\displaystyle x +2y - z = 2 \\
\displaystyle 2x -3y + 5z = 11\\
\displaystyle x -5y + 6z = 9\\
\end{array}
\right.\\
$
   \\    
   \begin{enumerate}
    \item é impossível;
    \item é possível e determinado;
    \item é possível e indeterminado;
    \item admite apenas a solução (1; 2; 3);
    \item admite a solução (2; 0; 0) 
    \end{enumerate}
    
    

    %%% RESP: D
    \item (FUVEST-2012) Em uma festa com $n$ pessoas, em um dado instante, 31 mulheres se retiraram e restaram convidados na razão de 2 homens para cada mulher. Um pouco mais tarde, 55 homens se retiraram e restaram, a seguir, convidados na razão de 3 mulheres para cada homem. O número $n$ de pessoas presentes inicialmente na festa era igual a
        \begin{enumerate}
		\item 100
		\item 105
		\item 115
        \item 130
		\item 135
		\end{enumerate}
    
    
    

    %%% RESP: D
    \item Considere o seguinte sistema de equações de incógnitas $x$ e $y$: \\
        $ 
\left\{
\begin{array}{ll}
\displaystyle 6x + 2y = 4 \\
\displaystyle 3x + 5y = 6\\
\displaystyle kx + 2y = 5\\
\end{array}
\right.\\
$
   \\        
   Para que valor de $k$ este sistema é possível e determinado ?
       \begin{enumerate}
    	\item 1
		\item 0
		\item -3
        \item 9
		\item 2
		\end{enumerate}



    %%% RESP: E
    \item Para qual valor de $m$  o sistema 
    $ 
\left\{
\begin{array}{ll}
\displaystyle x + y = 1 \\
\displaystyle 3x + 3y = m + 1\\
\end{array}
\right.\\
$
   \\
   é possível e indeterminado ?
    \begin{enumerate}
        \item 5
		\item -3
		\item 3
        \item -2
		\item 2
		\end{enumerate}


  %%% RESP: B
  
    \item (Mackenzie - 2008) O diretor de uma empresa, o Dr. Antonio, convocou todos os seus funcionários para uma reunião. Com a chegada do Dr. Antonio à sala de reuniões, o número de homens presentes na sala ficou quatro vezes maior que o número de mulheres também presentes na sala. Se o Dr. Antonio não fosse à reunião e enviasse sua secretária, o número de mulheres ficaria a terça parte do número de homens. A quantidade de pessoas, presentes na sala, aguardando o Dr. Antonio é
     \begin{enumerate}
    	\item 20
		\item 19
		\item 18
		\item 15
        \item 14
		\end{enumerate}
    
      
   
   %%% RESP: D     
    \item Para quais valores de $m$ e $n$ os dois sistemas a seguir são equivalentes? (Lembre-se, dois sistemas são equivalentes se possuem as mesmas soluções). \\
    $ 
\left\{
\begin{array}{ll}
\displaystyle x - y = 1 \\
\displaystyle 2x + y = 5\\
\end{array}
\right.\\
$
   \\
   
   $ 
\left\{
\begin{array}{ll}
\displaystyle mx -ny = -1 \\
\displaystyle nx + my = 2\\
\end{array}
\right.\\
$
   \\
    \begin{enumerate}
        \item m = 10 e n = 1
    	\item m = 0 e n = 10
		\item m = 1 e n = 10
		\item m = 0 e n = 1
        \item m = 1 e n = 0
		\end{enumerate}
        
    
    
    %%% RESP: A
    
    \item (UNESP - 2012) Em uma sala, havia certo número de jovens. Quando Paulo chegou, o número de rapazes presentes na sala ficou o triplo do número de garotas. Se, ao invés de Paulo, tivesse entrado na sala Alice, o número de garotas ficaria a metade do número de rapazes. O número de jovens que estavam inicialmente na sala (antes de Paulo chegar) era

     \begin{enumerate}
        \item 11
		\item 9
		\item 8
		\item 6
        \item 5
		\end{enumerate}
        
        
    
    %%% RESP: Amélia, Lúcia e Maria possuem, nessa ordem, R$ 24,00, R$ 18,00 e R$ 36,00.
            
    \item (FUVEST - 2007 - Segunda fase) Se Amélia der R\$3,00 a Lúcia, então ambas ficarão com a mesma quantia. Se Maria der um terço do que tem a Lúcia, então esta ficará com R\$6,00 a mais do que Amélia. Se Amélia perder a metade do que tem, ficará com uma quantia igual a um terço do que possui Maria. Quanto possui cada uma das meninas Amélia, Lúcia e Maria?
        
 
 

\end{enumerate}
\end{document}
