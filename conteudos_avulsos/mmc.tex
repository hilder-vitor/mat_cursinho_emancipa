
%% %% %% %% %% %% %% %% %% %% %% %% %% %% %% %% %% %%
%%%%%%%%%%%%%%%%%%%%%%%%%%%%%%%%%%%%%%%%%%%%%%%%%%%%%
%%%%%%%%    Cap�tulo sobre potencia��o		%%%%%
%%%%%%%%%%%%%%%%%%%%%%%%%%%%%%%%%%%%%%%%%%%%%%%%%%%%%
%%%%%%%%%%%%%%%		%%%%		%%%%%%%%%%%%%

\documentclass[a4paper,12pt]{article}
\usepackage[brazil]{babel}
\usepackage[latin1]{inputenc}

\usepackage{amsthm}
\usepackage{amssymb}  % para o simbolo dos reais e outros
% --------  Fim da se��o de par�metros globais

\begin{document}	% Come�o do conte�do.

\title{Matem�tica}
\author{Vitor Lima (vitor\_lp@yahoo.com)}
\maketitle




\section{MMC}
	Dizemos que um n�mero � um m�ltiplo de outro se ele pode ser escrito como um produto (uma multiplica��o) envolvendo este outro n�mero. \\
	Por exemplo, o n�mero 18 � m�ltiplo de 6, pois ele pode ser escrito como $6 \cdot 3$. Al�m disso, 18 tamb�m � um m�ltiplo de 9, pois $9 \cdot 2 = 18$. \\
	Ou seja, para saber se um n�mero qualquer, digamos, $x$, � m�ltiplo de um outro n�mero, digamos $y$, basta procurar por $x$ na tabuada do $y$ (quer dizer, procurar algum valor que possamos multiplicar por $y$ e obter $x$). \\


%%%%%%%%%%%%%%%%%%%%%%%%%%%%%%%%%%%%%%%%%%%
%%%%	Se��o de exerc�cios
%%%%%%%%%%%%

\section {Exerc�cios}

\begin{enumerate}
% exerc�cio um
\item Calcule as seguintes express�es envolvendo pot�ncias:
	\begin{description}	
	\item[a.] $3^1 + 3^{-1} - \frac{3}{9}$
	\item[b.] $\frac{2^{3+2}}{2^3} \cdot \frac{1}{2} $
	\item[c.] $\frac{5^{25}}{125 \cdot 125 \cdot 5^{20}} \cdot 5 - 17923^0$
	\item[d.] $(-2)^3 \cdot (\frac{3}{2})^2$
	\item[e.] $\frac{1}{3^{-3}}$
	\item[f.] $\frac{3^2 \cdot 4^4}{2^8 \cdot 9^1} - \frac{6^3}{2^3 \cdot 3^3}$
	\item[g.] $\frac{1978^1}{235817^0} - \frac{1978^5}{1978^2 \cdot 1978^2}$
	\item[h.] $(2^5 \cdot 2^{-3}) \cdot 2^{-5} \cdot (-2)^2$
	\item[i.] $\frac{(3 + 2)^5}{5^3} - 10^1 \cdot 10^0 - 5^2 + 25$
	\item[j.] $\left[\left(- \frac{2}{3} \right)^3 \cdot \left(\frac{6}{2} \right)^{-3} \cdot \left(\frac{2}{6} \right)^{-6} \cdot (-1) \right]^{1978}$
	\item[k.] $3^2 \cdot \frac{(7^9 + 6 \cdot 7^9)}{7^8} \cdot (\frac{1}{7}) ^ 2 \cdot 3^{-1} \cdot \frac{1}{7^{-1}} $
	\item[l.] $\left( 2^{3^2} \cdot 2 -\frac{1024^{10}}{(2^{10})^9} \right) \cdot 67$
	\end{description}
% exerc�cio 2
\item Em um tempo remoto, um s�bio salvou um rei usando de sua intelig�ncia para montar uma �tima estrat�gia de guerra. O rei, para agradecer este feito, disse ao s�bio que lhe daria o que quisesse. Ent�o, o s�bio pediu ao rei que pegasse um tabuleiro de xadrez e colocasse dois gramas de ouro na primeira casa, quatro na segunda, oito na terceira, e assim sucessivamente, sempre dobrando a quantidade, e no fim desse a ele a quantidade de ouro que estivesse no tabuleiro.
\\ O rei achou muito pouco, mas resolveu atender o pedido.
\\ Voc� acha que o rei conseguiu fazer o que o s�bio pediu? Por que?
\\ (Observa��o: um tabuleiro de xadrez tem 64 casas).


\end{document}
